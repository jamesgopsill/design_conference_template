\documentclass{design}

\usepackage{siunitx}
\usepackage{hanging}
\addbibresource{instructions_for_authors.bib}

\begin{document}

{
    \titlefont
    \small
    \begin{wrapfigure}{r}{0.2\textwidth}
        \raggedleft
        \vspace{-0.2cm}
        \includegraphics[width=1.8cm]{design-logo.png}
    \end{wrapfigure}
    \noindent \textbf{INTERNATIONAL DESIGN CONFERENCE - DESIGN 2020}\\
    https://doi.org/10\ldots{}
    
    \vspace{2cm}
    
    \Large \noindent \textbf{Instructions for Authors}
    
    \vspace{1cm}
    
    \normalsize \noindent [Authors will be inserted automatically]\\[0.2cm]
    \footnotesize \noindent [Institutions will be inserted automatically]\\[0.1cm]
    \footnotesize \noindent [Corresponding author data will be inserted automatically]\\
    
    \begin{mdframed}[backgroundcolor=gray!20] 
        \normalsize \noindent \textbf{Abstract} \\
        \normalfont \small [Abstract will be inserted automatically] \\
        \normalfont \small [-\phantom{}-\phantom{}-] \\
        \normalfont \small [-\phantom{}-\phantom{}-] \\
        \normalfont \small [-\phantom{}-\phantom{}-] \\
        \normalfont \small [-\phantom{}-\phantom{}-] \\
        \normalfont \small [-\phantom{}-\phantom{}-] \\
        \normalfont \small [-\phantom{}-\phantom{}-]
    \end{mdframed}
    
    \small \noindent \textit{[Keywords will be inserted automatically]}
    
    \vspace{0.5cm}
}


\section{Section}

This document is intended to guide contributors when preparing submissions for the 18th International Design Conference – DESIGN 2024. We kindly ask you to follow the enclosed instructions carefully. The electronic proceedings will be prepared using MS Word. It is therefore critical for us to receive properly formatted papers based on the \textbf{MS Word template} available for download on the conference website (http://www.designconference.org). This template is for both the first and camera-ready submissions. The first submission of the papers (prior to the reviews) must be uploaded as a PDF file only, whereas the camera-ready papers should be uploaded as both PDF and DOCX files.
Please delete all comments within the manuscript before submission. All changes in the document must be accepted or rejected before the submission. The submitted paper should also not contain any markup.

Note that the papers should be submitted exclusively using the ConfTool conference management system as instructed on the Design Conference website. During the first step of the \textbf{ConfTool} submission procedure, the authors will be asked to submit an abstract (max 600 characters) and include between three and five keywords. A minimum of three keywords must be selected from the predefined list and up to two may be of your choice.

In order to produce conference proceedings of a professional and consistent quality, the template is protected and cannot be modified. If the document is based on another template or if this template is changed, the paper will be returned to the authors for correction and reformatting.

\section{Paper format}

The page size should be A4 (210 mm or 8.27-inch x 297 mm or 11.69-inch) with margins as specified: 25 mm left, 25 mm right, 25 mm top, and 32 mm bottom (this will give a text area of 160 mm width and 240 mm height). \textbf{The length of the paper is limited to 10 (ten) pages}. No additional pages will be accepted. Due to layout reasons, the number of pages must be \textbf{even}. Custom headers or footers are not allowed. Please do not add page numbers.

\section{Automatically inserted data}

Please note that authors’ names, institutions, contact details, keywords, and abstract will be inserted automatically from the ConfTool system during the preparation of the proceedings. Please, do not edit or delete these placeholders in the template.

We advise authors to follow the instructions provided within ConfTool and ensure that multiple instances (for example names and institutions of authors contributing to more than one paper) are written in the same way for all occurrences.

\section{Formatting requirements}

When formatting your paper, please use the predefined Design Conference (DC) styles that are specified within the template. This “Instructions for authors” document was prepared using the Design Conference template and can be used as an example. The predefined style names start with the “DC\_” characters which are followed by the name of the style (e.g. “DC\_Normal” for main text paragraphs, “DC\_Heading1” for first level headings, “DC\_Figure\_Caption” for figure captions, etc.). The overview of properties for each style is available in Appendix B of this document.

\subsection{Title}

The title of the paper must be written \textbf{manually}, at the top of the first page. The title should be in “DC\_Title” style (uppercase) and may not end in a full stop.

\subsection{Headings}

All headings should be written in sentence case (cap on the first word only). First level headings should be in “DC\_Heading1” style, second level headings should be in “DC\_Heading2” style, and third level headings should be in “DC\_Heading3” style. These three predefined heading styles are automatically numbered. Please do not use headings levels four or lower.

Unnumbered headings (e.g. Acknowledgement, References and Appendix headings) should be in \\ “DC\_Unnumbered\_Heading” style. Headings may not end in a full stop.

\subsection{Main Text}

Main text should be in “DC\_Normal” style. Please do not use blank lines between paragraphs of main text. For highlighting characters in bold, please use the “DC\_Char\_Bold” style. For italic, use the “DC\_Char\_Italic” style, and for bold and italic use “DC\_Char\_BoldItalic”.

SI units should be used throughout the paper (e.g. \SI{100}{\milli\metre}, \SI{10}{\kilo\newton}, \SI{30}{\newton\per\milli\metre\squared}).

\subsection{Lists}

Three list styles have been predefined in the template: bulleted list, numbered list and lettered list. It is suggested not to use more than three list levels.

\subsubsection{Bulleted lists}

\begin{itemize}
    \item Bulleted (unnumbered) lists should use bullets as shown here.
    \item Use the style “DC\_List\_Bulleted” for bulleted lists.
    \item First level
    \begin{itemize}
        \item Second level
        \begin{itemize}
            \item Third level
        \end{itemize}
    \end{itemize}
\end{itemize}

\subsubsection{Numbered lists}

\begin{enumerate}
    \item Numbered lists should start with an ordinal number as shown here.
    \item Use the style “DC\_List\_Numbered” for numbered lists with automatic numbering.
    \item First level
    \begin{enumerate}
        \item Second level
        \begin{enumerate}
            \item Third level
        \end{enumerate}
    \end{enumerate}
\end{enumerate}


\subsubsection{Lettered lists}

\begin{letterlist}
    \item Lettered lists use lower case letters in an alphabetic order.
    \item Use the style “DC\_List\_Lettered” for automatic lettered lists.
    \item First level
    \begin{letterlist}
        \item Second level
        \begin{letterlist}
            \item Third level
        \end{letterlist}
    \end{letterlist}
\end{letterlist}

Figures should be in line with text and centred on the page using the “DC\_Figure” style. Please do not place text boxes over the figures. It is strongly recommended to insert the figures into the document from an external file using the “Insert->Pictures from file” menu option. Inserted figures should be of the highest possible quality. Please avoid insertion of the graphical object as figures.

Figures should be cited within the main text before their appearance in the paper. An example of inserting figures and figure captions can be seen below (\cref{fig:one}).

\begin{figure}[h]
    \centering
    \includegraphics[width=0.2\textwidth]{design-logo.png}
    \caption{Figure Caption}
    \label{fig:one}
\end{figure}

Figure captions should be labelled with “Figure” followed by the ordinal number of the figure. Figure captions must be placed below the figure in “DC\_Figure\_Caption” style. Within a caption, use soft returns only (i.e. manual line breaks, “Shift-Enter”). Figure captions may not end in a full stop.

\subsection{Tables}

Table captions should be labelled with “Table” followed by the ordinal number of the table. Table captions must be placed above the table in “DC\_Table\_Caption” style and may not end in a full stop. Within a caption, use soft returns only (i.e. manual line breaks, “Shift-Enter”).

Tables should be centred on the page, whereas the text inside the table cells can be left-aligned (style: “DC\_Table\_Left”), right-aligned (style: “DC\_Table\_Right”), centred (style: “DC\_Table\_Center”), or justified (style: “DC\_Table\_Justified”). It is recommended that the tables do not spread over multiple pages. Tables should be cited within the main text before their appearance in the paper. An example of using tables and table captions can be seen below \pref{tab:one}.

\begin{table}[h!]
    \centering
    \caption{Table Caption}
    \label{tab:one}
    \begin{tabular}{ |l|c|r| }
        \hline
        Table text left & Table text middle & Table text right \\
        \hline
        \multicolumn{3}{|l|}{Table text justified} \\
        \hline
        \multicolumn{3}{|l|}{Use character styles to make table text bold (\textbf{DC\_Char\_Bold}) or italic (\textit{DC\_Char\_Italic})} \\
        \hline
        \multicolumn{3}{l}{Use “DC\_Table\_Left” style for table legends}
    \end{tabular}
\end{table}

If a text paragraph follows the table, please leave one blank line between the table and the paragraph. Blank lines are not necessary in cases where the table is followed by a heading, figure, or another table.

\subsection{Equations}

Equations should be left aligned with \SI{1.27}{\centi\metre} left indentation and contain an equation number at the right margin in “DC\_Equation” style. Use tab key to move from the equation to the equation number. Equation numbers must be entered manually. Equations should be cited within the main text before their appearance in the paper. An example of using equations can be seen below \pref{equ:one}.
\begin{equation}
    x^2y^2 + Ax_2 - 2Bxy = 0
    \label{equ:one}
\end{equation}

Subscript and superscript characters within the text can be added using the “DC\_Char\_Subscript” and “DC\_Char\_Superscript” styles.

\subsection{Direct quotes}

Reporting of the exact words of an author or speaker (direct quotes) must be placed inside quotation marks and formatted using the “DC\_Direct\_Quote” style. For example:

\begin{displayquote}\it
“Lorem ipsum dolor sit amet, consectetur adipiscing elit, sed do eiusmod tempor incididunt ut labore et dolore magna aliqua. Ut enim ad minim veniam, quis nostrud exercitation ullamco laboris nisi ut aliquip ex ea commodo consequat.”
\end{displayquote}

When used in-text, the direct quotes may stay in “DC\_Normal” paragraph style and formatted using the “DC\_Char\_Italic” character style, for example: “\textit{Duis aute irure dolor in reprehenderit in voluptate velit esse cillum dolore eu fugiat nulla pariatur}”.

\subsection{Footnotes}

Please avoid using footnotes. However, if footnotes are placed\footnote{Use “References->Insert Footnote” menu option to automatically place a footnote.}, the footnote text must be formatted  using the “DC\_Footnote” style.

\subsection*{Acnkowledgement}

Acknowledgement(s) (if any) should be in “DC\_Acknowledgement style”. Acknowledgement should be placed before references, with the title in “DC\_Unnumbered\_Heading” style.

\subsection{References}

The references heading should be in “DC\_Unnumbered\_Heading” style. Citations to published work throughout the paper should follow the Harvard-based notation, as defined within these instructions. For the use of citation tools, the following styles may be useful to some extent:

\begin{itemize}
    \item Citavi: Harvard (Emerald)
    \item Endnote: Harvard UL
    \item Mendeley: Emerald - Harvard
\end{itemize}

The main text should include references using the surname(s) of the author(s) and year. For in-text citations please follow: (Author, year of publication) / (Author1 and Author2, year of publication) / (Author1 et al., year of publication), or Author (year of publication) / Author1 and Author2 (year of publication) / Author1 et al. (year of publication). Use a semicolon (;) for separating multiple references inside brackets, for example (Hubka and Eder, 1992; Pahl and Beitz, 1996). If the year of publication is not available, it must be replaced with “n.d.”, (Author, n.d.).

A portion of a paper might thus contain sentences such as: “This work is grounded in systematic approaches to design (Pahl and Beitz, 1996), building on the work of Jensen and Andreasen (2010), and the affordance-based relational theory developed by Maier and Fadel (2009a, 2009b). Chakrabarti et al. (2011) present an overview of computer-based design synthesis research”.
Each reference needs to include the \textbf{Digital Object Identifier (DOI)} if the publication has one. DOI can be found on the reference source website or in the CrossRef database (http://www.crossref.org). 

Please ensure that every reference cited in the text is also included in the reference list (and vice versa). The list of references must be sorted in alphabetical order. Note that if the publication has more than six authors, only the first five are listed, followed by “et al.” after the fifth author’s name.

Please use the following guidelines when formatting the list of references:

\textbf{\textit{Books}}

\begin{hangparas}{0.63cm}{1}
\textbf{Author, A.A., Author, B.B. and Author, C.C. (year of publication), Title of Book, Publisher, Place of publication. DOI (if present)}

\fullcite{beitz1996}\\[-0.5em]

\end{hangparas}

\textbf{\textit{Book chapters}}

\begin{hangparas}{0.63cm}{1}
\textbf{Author of chapter, A.A., Author of chapter, B.B. and Author of chapter C.C. (year of publication), “Title of chapter”, In: Editor, A.A., Editor, B.B. and Editor, C.C. (Eds.), Title of book, Publisher, Place of publication, pp. (insert page numbers). DOI (if present)}

\fullcite{finger2002}\\[-0.5em]
\end{hangparas}

\textbf{\textit{Journals}}

\begin{hangparas}{0.63cm}{1}
\textbf{Author, A.A., Author, B.B. and Author, C.C. (year of publication), “Title of Article”, Title of Journal, Vol. (insert volume number) No. (insert issue number), pp. (insert page numbers). DOI (if present)}

\fullcite{cash2013}\\[-0.5em]

\end{hangparas}

\textbf{\textit{Conference Papers}}

\begin{hangparas}{0.63cm}{1}
\textbf{Author, A.A., Author, B.B. and Author, C.C. (year of publication), “Title of Paper”, Title of published proceeding (which may include place and date(s) held), Publisher, Place of publication, pp. (insert page numbers). DOI (if present)}

\fullcite{gopsill2021}\\[-0.5em]

\end{hangparas}

\textbf{\textit{Electronic Sources}}

\begin{hangparas}{0.63cm}{1}
\textbf{Author, A.A., Author, B.B. and Author, C.C. (year of publication/last updated), Title. [online] Publisher. Available at: (insert URL without hyperlinks) (accessed date). DOI (if present)}

\fullcite{martin2017}\\[-0.5em]

\end{hangparas}

\textbf{\textit{Reports}}

\begin{hangparas}{0.63cm}{1}
\textbf{Organisation/Author (year of publication), Title of Report, Publisher, Place of publication.}

\fullcite{ds2012}\\[-0.5em]

\end{hangparas}

\textbf{\textit{Thesis (PhD thesis, Master thesis)}}

\begin{hangparas}{0.63cm}{1}
\textbf{Thesis: Author, A. (year of publication), Title of Thesis, [Designation], Awarding institution. DOI (if present)}

\fullcite{wynn2007}\\[-0.5em]

\end{hangparas}


\textbf{\textit{Standards}}

\begin{hangparas}{0.63cm}{1}
\textbf{Organisation/Author (year of publication), Standard title, Publisher, Place of publication.}

\fullcite{iso9000}
\end{hangparas}


\section*{Appendix A}

Appendices (if any) should be in “DC\_Normal” style. They should be placed after references, with the title in “DC\_Unnumbered\_Heading” style and labelled using letters (e.g. Appendix A, Appendix B, Appendix C, etc.). Within the text, appendices must be referenced in the same way, e.g. “Please consult Appendix A for the complete list of questions”. Appendices are included in the 10-page limit.

\section*{Appendix B: Style Overview}

\cref{tab:two,tab:three} give an overview of properties for each style to be used for the preparation of the Design Conference papers. Be aware that these styles are already built into the paper template.

\begin{table}[h!]
    \centering
    \small
    \caption{Overview of Design Conference paragraph styles}
    \label{tab:two}
    \begin{tabular}{ |l|L{0.7\textwidth}| }
        \hline
        \textbf{Style name} & \textbf{Style properties} \\
        \hline
        DC\_Normal &Times New Roman, regular, 11 pt, justified, single line, no spacing \\ \hline
        DC\_Heading & Trebuchet MS, bold, 14 pt, left aligned, single line, 12 pt space before and 3 pt space after \\ \hline
        DC\_Heading2 & Trebuchet MS, bold, 12 pt, left aligned, single line, 12 pt space before and 3 pt space after \\ \hline
        DC\_Heading3 & Trebuchet MS, italic, 11 pt, left aligned, single line, 12 pt space before and 3 pt space after \\ \hline
        DC\_Unnumbered\_Heading & Trebuchet MS, bold, 12 pt, left aligned, single line, 12 pt space before and 3 pt space after \\ \hline
        DC\_List\_Bulleted & Times New Roman, regular, 11 pt, justified, single line, 6 pt space after changing to other style, 0.63 cm hanging indent and 0.63 cm tabs, dot bullet \\ \hline
        DC\_List\_Numbered & Times New Roman, regular, 11 pt, justified, single line, 6 pt space after changing to other style, 0.63 cm hanging indent and 0.63 cm tabs, automatic numbering \\ \hline
        DC\_List\_Lettered & Times New Roman, regular, 11 pt, justified, single line, 6 pt space after changing to other style, 0.63 cm hanging indent and 0.63 cm tabs, automatic lettering \\ \hline
        DC\_Figure & Times New Roman, regular, 11 pt, centred, single line, 12 pt spacing before \\ \hline
        DC\_Figure\_Caption & Trebuchet MS, bold, 10 pt, centred, single line, 3 pt space before and 6 pt space after \\ \hline
        DC\_Table\_Caption & Trebuchet MS, bold, 10 pt, centred, single line, 12 pt space before and 3 pt space after \\ \hline
        DC\_Table\_Left & Times New Roman, regular, 10 pt, left aligned, single line, 1 pt space before and 1 pt space after \\ \hline
        DC\_Table\_Center & Times New Roman, regular, 10 pt, centred, single line, 1 pt space before and 1 pt space after \\ \hline
        DC\_Table\_Right & Times New Roman, regular, 10 pt, right aligned, single line, 1 pt space before and 1 pt space after \\ \hline
        DC\_Table\_Justified & Times New Roman, regular, 10 pt, justified, single line, 1 pt space before and 1 pt space after \\ \hline
        DC\_Equation & Times New Roman, regular, 10 pt, left aligned, single line, 6 pt space before and 6 pt space after \\ \hline
        DC\_Direct\_Quote & Times New Roman, italic, 11 pt, justified, single line, no spacing, 6 pt space before and 6 pt space after, 1.26 cm hanging indent and 1.26 cm tabs \\ \hline
        DC\_Acknowledgement & Times New Roman, regular, 10 pt, justified, single line, no spacing \\ \hline
        DC\_References & Times New Roman, regular, 10 pt, justified, single line, no spacing, 0.63 cm hanging indent \\ \hline
        DC\_Footnote & Times New Roman, regular, 10 pt, justified, single line, no spacing \\ \hline
        DC\_Title & Trebuchet MS, bold, 15 pt, left aligned, single line, 90 pt space before and 24 pt space after, uppercase \\ \hline
    \end{tabular}
\end{table}

\begin{table}[h!]
    \centering
    \small
    \caption{Overview of Design Conference character styles}
    \label{tab:three}
    \begin{tabular}{ |l|L{0.7\textwidth}| }
        \hline
        \textbf{Style name} & \textbf{Style properties} \\
        \hline
        DC\_Char\_Bold & Default paragraph font, bold \\ \hline
        DC\_Char\_Italic & Default paragraph font, italic \\ \hline
        DC\_Char\_BoldItalic & Default paragraph font, bold italic \\ \hline
        DC\_Char\_Subscript & Default paragraph font, subscript \\ \hline
        DC\_Char\_Superscript & Default paragraph font, superscript \\ \hline
    \end{tabular}
\end{table}

\end{document}
